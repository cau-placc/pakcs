\section{External Operations}
\label{sec-external-operations}

\index{function!external}\index{operation!external}\index{external operation}
An \emph{external operation} is an operation which have no defined rules
in a Curry program.
Instead, such an operation must be declared as \code{external}
in the Curry source code
and an implementation for this external operation
must be inserted in the corresponding back end.
In this section we describe how external operations
can be implemented in \CYS.

In general, an external operation is defined as follows
in the Curry source code:
\begin{enumerate}
\item
Provide a type declaration for the external operation somewhere
in the body of the appropriate Curry file.
Note that external operations should not be overloaded,
i.e., the type declaration should not contain any type class constraint.
\item
For external operations it is not allowed to define any
rule since their semantics is determined by an external implementation.
Instead of the defining rules, one has to write
\begin{curry}
f external
\end{curry}
somewhere in the file containing the type declaration for 
the external operation \code{f}.
\end{enumerate}
For instance, the addition on integers can be declared as
an external operation as follows:
\begin{curry}
(+) :: Int -> Int -> Int
(+) external
\end{curry}
Since \CYS compiles Curry programs into Prolog programs,
the actual implementation of an external operation
must be contained in some Prolog code that
is added to the compiled code by \CYS.
This can be done as follows:
%
\begin{enumerate}
\item
The Prolog code implementing the external operations declared
in module \code{M} must be put into the Prolog file \code{M.pakcs.pl}.
This file must be stored in the directory
containing the source code of the corresponding Curry module.
The contents of this file will be automatically added to the
compiler Curry program.

\item
In the general case (see below for exceptions),
the \CYS compiler generates a \emph{standard interface} to
external operations so that an $n$-ary operation is
implemented by an $(n+1)$-ary predicate where the last argument must
be instantiated to the result of evaluating the operation.
If \code{M.f} is the qualified name of the
external operation \code{f} defined in module \code{M},
then the predicate implementing this operation
must have the name \code{'M.f'} (note that this name
must be enclosed in ticks in Prolog).
The standard interface passes all arguments in their current form
to the predicate, i.e., it can be used if it is ensured that
all arguments are fully evaluated.
For the operation \code{(+)} shown above, this might not be the
case: in a call like \ccode{fac 4 + 3 * 7}, both arguments
mube be evaluated to some number before the external code
for the addition is called.
This can be ensured by enforcing the evaluation of the arguments
before calling the actual external operation.
For instance, the external operation for adding two integers
requires that both arguments must be evaluated to
a non-variable head normal form
(which is identical to the ground constructor normal form). Therefore,
the operation \ccode{+} can be implemented in the prelude by
\begin{curry}
(+)   :: Int -> Int -> Int
x + y = (prim_plusInt $\$$# y) $\$$# x

prim_plusInt :: Int -> Int -> Int
prim_plusInt external
\end{curry}
where \code{prim_plusInt} is the actual external operation implementing
the addition on integers.
Hence, the Prolog code implementing \code{prim_plusInt} can be as
follows (note that the arguments of \code{(+)} are passed in reverse
order to \code{prim_plusInt} in order to ensure a left-to-right
evaluation of the original arguments by the calls to \code{(\$\#)}):
\begin{curry}
'Prelude.prim_plusInt'(Y,X,R) :- R is X+Y.
\end{curry}

\item
The \emph{standard interface for I/O actions}, i.e., external operations
with result type \code{IO~a}, assumes that the I/O action
is implemented as a predicate (with a possible side effect)
that instantiates the last argument to the returned value of type \ccode{a}.
For instance, the primitive predicate \code{prim_getChar}
implementing the prelude I/O action \code{getChar}
can be implemented by the Prolog code
\begin{curry}
'Prelude.getChar'(C) :- get_code(N), char_int(C,N).
\end{curry}
where \code{char_int} is a predicate (from the \CYS run-time system)
relating the internal Curry representation of a character
with its ASCII value.

\item
If some arguments passed to the external operations are not fully evaluated
or the external operation might suspend, the implementation must follow
the structure of the \CYS run-time system by using the \emph{raw interface}
instead of the standard interface.
For this purpose, it is necessary to tell \CYS about the
non-standard interface.
Thus, if the Curry module \code{Mod} contains external operations
where the standard interface should not be used,
there must be a file named \code{Mod.pakcs} containing the
specification of these external operations. The contents of this file
is in XML format and has the following general structure:\footnote{%
\url{http://www.curry-lang.org/pakcs/primitives.dtd} contains a DTD
describing the exact structure of these files.}
\begin{curry}
<primitives>
  $\textit{specification of external operation~}f_1$
  $\ldots$
  $\textit{specification of external operation~}f_n$
</primitives>
\end{curry}
The specification of an external operation $f$
with arity $n$ has the form
\begin{curry}
<primitive name="$f$" arity="$n$">
  <entry>pred[raw]</entry>
</primitive>
\end{curry}
where \code{pred} is the name of a predicate
implementing this operation. Note that the operation $f$ must be
declared in module \code{Mod}: either as an external operation
or defined in Curry by equations. In the latter case,
the Curry definition is not translated but calls to this operation
are redirected to the Prolog code specified above.

Furthermore, the list of specifications can also contain entries of the form
\begin{curry}
<ignore name="$f$" arity="$n$" />
\end{curry}
for operations $f$ with arity $n$ that are declared in module \code{Mod}
but should be ignored for code generation, e.g., since they are
never called w.r.t.\ to the current implementation of external operations.
For instance, this is useful when operations that can
be defined in Curry should be (usually more efficiently) are implemented
as external operations.

The suffix \ccode{[raw]} used above indicates that the corresponding Prolog
code follows the structure of the \CYS compilation scheme.
For instance, if we want to use the raw interface for the external operation
\code{prim_plusInt},
the specification file \code{Prelude.pakcs} must have an entry of the form
\begin{curry}
<primitive name="prim_plusInt" arity="2">
  <entry>prim_plusInt[raw]</entry>
</primitive>
\end{curry}
In the raw interface, the actual implementation of
an $n$-ary external operation consists
of the definition of an $(n+3)$-ary predicate $pred$.
The first $n$ arguments are the corresponding actual arguments.
The $(n+1)$-th argument is a free variable which must be
instantiated to the result of the operation call after
successful execution. The last two arguments
control the suspension behavior of the operation
(see \cite{AntoyHanus00FROCOS} for more details):
The code for the predicate $pred$
should only be executed when the $(n+2)$-th argument
is not free, i.e., this predicate has always the
SICStus-Prolog block declaration
\begin{curry}
?- block $pred$(?,$\ldots$,?,-,?).
\end{curry}
In addition, typical external operations should suspend
until the actual arguments are instantiated. This can be ensured
by a call to \code{ensureNotFree} or \code{(\$\#)}
before calling the external operation. Finally, the
last argument (which is a free variable at call time)
must be unified with the $(n+2)$-th argument
after the operation call is successfully evaluated
(and does not suspend). Additionally, the actual (evaluated) arguments
must be dereferenced before they are accessed.
Thus, an implementation
of the external operation for adding integers is as follows in the raw interface:
\begin{curry}
?- block prim_plusInt(?,?,?,-,?).
prim_plusInt(RY,RX,Result,E0,E) :-
     deref(RX,X), deref(RY,Y), Result is X+Y, E0=E.
\end{curry}
Here, \code{deref} is a predefined predicate for dereferencing the
actual argument into a constant (and \code{derefAll} for dereferencing
complex structures).
\end{enumerate}
%
Note that arbitrary operations implemented in C or Java can be connected to
\CYS by using the corresponding interfaces of the underlying Prolog system.


%%% Local Variables: 
%%% mode: latex
%%% TeX-master: "manual"
%%% End: 
