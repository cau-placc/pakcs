\section{Technical Problems}

Due to the fact that Curry is intended to implement
distributed systems (see Appendix~\ref{sec-ports}),
it might be possible that some technical problems
arise due to the use of sockets for implementing these
features. Therefore, this section gives some information
about the technical requirements of \CYS and how to solve
problems due to these requirements.

There is one fixed port that is used by the implementation of \CYS:
\begin{description}
\item[Port 8766:] This port is used by the
{\bf Curry Port Name Server} (CPNS) to implement symbolic names for
ports in Curry (see Appendix~\ref{sec-ports}).
If some other process uses this port on the machine,
the distribution facilities defined in the module \code{Ports}
(see Appendix~\ref{sec-ports}) cannot be used.
\end{description}
If these features do not work, you can try to find out
whether this port is in use by the shell command
\ccode{netstat -a | fgrep 8766} (or similar).

The CPNS is implemented as a demon listening on its port 8766
in order to serve requests about registering a new symbolic
name for a Curry port or asking the physical port number
of a Curry port. The demon will be automatically started for
the first time on a machine when a user compiles a program
using Curry ports. It can also be manually started and terminated by the
scripts \code{\cyshome/cpns/start} and
\code{\cyshome/cpns/stop}.
If the demon is already running, the command \code{\cyshome/cpns/start}
does nothing (so it can be always executed
before invoking a Curry program using ports).

If you detect any further technical problem,
please write to
\begin{center}
\code{mh@informatik.uni-kiel.de}
\end{center}

%%% Local Variables: 
%%% mode: latex
%%% TeX-master: "manual"
%%% End: 
